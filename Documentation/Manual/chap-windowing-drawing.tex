\chapter{Windowing system drawing functions}

A typical windowing system provides a hierarchy of rectangular areas
called windows.  When a drawing functions is called to draw an object
(such as a line or a circle) in a window of such a hierarchy, the
arguments to the drawing function will include at least the window and a
number of coordinates relative to (usually) the upper left corner of the
window.

To translate such a request to the actual altering of pixel values in
the video memory, the windowing system must translate the coordinates
given as argument to the drawing functions into coordinates relative to
the upper left corner of the entire screen.  This is done by a
composition of translation transformations applied to the initial
coordinates.  These transformations correspond to the position of each
window in the coordinate system of its parent.

Thus a window in such a system is really just some values indicating its
height, its width, and its position in the coordinate system of its
parent, and of course information about background and foreground colors
and such.
